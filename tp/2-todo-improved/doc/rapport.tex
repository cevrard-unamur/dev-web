\documentclass{article}

% Formatting
\usepackage[utf8]{inputenc}
\usepackage[margin=1in]{geometry}
\usepackage[titletoc,title]{appendix}
\usepackage{hyperref}

% Title content
\title{Architecture des applications web - TP 1}
\author{Cédric Evrard}
\date{10 avril 2020}

\begin{document}
\maketitle

Le fichier d'entré de l'application se nomme $flask\_engine.py$. Un fichier $bash$ nommé $run-dev.sh$ permet de lancer l'application en mode développement directement, sans avoir à configurer de variable spécifique.

\section{Utilisation de Flask}
L'utilisation de Flask-WTF permet de facilement créer et gérer des formulaires, sans avoir à jouer avec l'objet $request.form$.

Ensuite, SQLAlchemy permet de facilement travailler avec la base de données, sans avoir à créer de requête SQL (et donc potentiellement devoir apprendre un autre langage).

\section{Apprentissage}
Les concepts de ce TP m'étant connu (gestion de formulaire, ORM), l'apprentissage principale était lié à l'utilisation de Flask et des éléments propres à celui-ci ou à ses packages.

\section{Difficultés}
Je n'ai pas eu de difficulté particulière pour cette deuxième partie. L'utilisation de Flask et Python était plus déjà plus simple grâce à la création de l'application pour le TP1.

Le seul petit problème à été l'intégration de Flask-Mail. Même si celui-ci est très simple, il existe deux documentations différentes pour des packages "Flask-Mail" (\hyperref[https://pythonhosted.org/Flask-Mail/]{https://pythonhosted.org/Flask-Mail/} et\\ \hyperref[https://pythonhosted.org/flask-mail/]{https://pythonhosted.org/flask-mail/}). Ce qui m'a posé quelques problèmes ayant commencé en utilisant la mauvaise documentation.

\section{Information}
Un compte \emph{SuperAdmin} existe et permet d'accéder à l'ensemble des fonctionnalités de l'application. Celui-ci est accessible via le login \textbf{admin} et le mot de passe \textbf{password}. Ce compte ne peut être bloqué ou supprimé, et on ne peut changer son statut de \emph{SuperAdmin}.

Un compte administrateur ne peut pas lui-même se supprimer ou se bloquer, et ne peut lui-même changer son statut d'administrateur.

Lors de l'inscription d'un utilisateur, un email est envoyé à l'adresse email qu'il a renseigné. Il n'y a pas de configuration particulière dans le fichier $config.py$ pour Flask-Mail car le serveur SMTP utilisé pour les tests était hébergé sur ma machine via \emph{mailslurper}.

\end{document}